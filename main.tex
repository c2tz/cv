\documentclass[11pt, a4paper]{article}

% ---- package declaration ---- %
\usepackage[ignoreheadfoot, top=1.0 cm, bottom=2.0 cm, left=2.0 cm, right=2.0 cm, footskip=1.0 cm]{geometry}
\usepackage[explicit]{titlesec}
\usepackage{tabularx}
\usepackage{array}
\usepackage[dvipsnames, svgnames]{xcolor}
\usepackage[pdftitle={CV - TAILLIEZ Clément}, pdfauthor={TAILLIEZ Clément}, pdfcreator={TAILLIEZ Clément}, colorlinks=true, urlcolor=primaryColor]{hyperref}
\usepackage[pscoord]{eso-pic}
\usepackage{enumitem}
\usepackage{fontawesome5}
\usepackage{amsmath}
\usepackage{calc}
\usepackage{bookmark}
\usepackage{lastpage}
\usepackage{changepage}
\usepackage{paracol}
\usepackage{ifthen}
\usepackage{needspace}
\usepackage{iftex}
\usepackage{datetime2}
\usepackage[sfdefault]{roboto}
\usepackage{luacode}

% ---- for accessibility ---- %
\ifPDFTeX
    \input{glyphtounicode}
    \pdfgentounicode=1
    \usepackage[T1]{fontenc}
    \usepackage[utf8]{inputenc}
    \usepackage{lmodern}
\fi

% ---- test base64 ---- %
\begin{luacode*}

local mime = require("mime")
local lft = lua.get_functions_table()
lft[#lft + 1] = function()
    local str = token.scan_string()
    tex.sprint(-2, (mime.b64(str)))
end
token.set_lua("BaseEncode", #lft, "global")
lft[#lft + 1] = function()
    local str = token.scan_string()
    tex.sprint(-2, (mime.unb64(str)))
end
token.set_lua("BaseDecode", #lft, "global")

\end{luacode*}

% ---- adjustment ---- %
\definecolor{primaryColor}{RGB}{0, 79, 144}
\AtBeginEnvironment{adjustwidth}{\partopsep0pt}
\pagestyle{empty}
\setcounter{secnumdepth}{0}
\setlength{\parindent}{0pt}
\setlength{\topskip}{0pt}
\setlength{\columnsep}{0.15cm}
\makeatletter

% ---- header env ---- %
\newenvironment{header}{\setlength{\topsep}{0pt}\par\kern\topsep\centering\color{primaryColor}\linespread{1.5}}{\par\kern\topsep}\let\hrefWithoutArrow\href

% ---- space for title ---- %
\titleformat{\section}{\needspace{4\baselineskip}\Large\color{primaryColor}}{}{}{\textbf{#1}\hspace{0.15cm}\titlerule[0.8pt]\hspace{-0.1cm}}[]
\titlespacing{\section}{-1pt}{0.3 cm}{0.2 cm}

% ---- highlights env ---- %
\newenvironment{highlights}{
    \begin{itemize}
        [topsep=0.10 cm, parsep=0.10 cm, partopsep=0pt, itemsep=0pt, leftmargin=0.4 cm + 10pt]}{
    \end{itemize}}
\newenvironment{highlightsforbulletentries}{
    \begin{itemize}
        [topsep=0.10 cm, parsep=0.10 cm, partopsep=0pt, itemsep=0pt, leftmargin=10pt]}{
    \end{itemize}}

% ---- column entry env ---- %
%1
\newenvironment{onecolentry}{
    \begin{adjustwidth}
        {0.2 cm + 0.00001 cm}{0.2 cm + 0.00001 cm}}{
    \end{adjustwidth}
}
%2
\newenvironment{twocolentry}[2][]{\onecolentry\def\secondColumn{#2}\setcolumnwidth{\fill, 4.5 cm}
    \begin{paracol}{2}}{
        \switchcolumn\raggedleft\secondColumn
    \end{paracol}
\endonecolentry
}
%3
\newenvironment{threecolentry}[3][]{\onecolentry\def\thirdColumn{#3}\setcolumnwidth{1 cm, \fill, 4.5 cm}
    \begin{paracol}{3}
        {\raggedright #2}\switchcolumn}{\switchcolumn\raggedleft\thirdColumn
    \end{paracol}
\endonecolentry
}

% ---- footer ---- %
\let\ps@customFooterStyle\ps@plain
\DTMsetstyle{default}
\DTMsetdatestyle{ddmmyyyy}
\DTMsetup{datesep=/,hourminsep=h,minsecsep=:}
\patchcmd{\ps@customFooterStyle}{\thepage}{
    \begin{minipage}
        {\textwidth}\centering\color{darkgray}
        \textrm{\footnotesize{Last update :}}
        \textrm{\footnotesize{\textit\date{\today} at \DTMcurrenttime{} ·}}
        \textrm{\footnotesize{\hrefWithoutArrow{https://github.com/c2tz/cv}{source code} [MIT License]}}
        \textrm{\footnotesize · p.\thepage{} on \pageref*{LastPage}}
    \end{minipage}}{}{}
\makeatother
\pagestyle{customFooterStyle}

% ---- href new cmd ---- %
\renewcommand{\href}[2]{\hrefWithoutArrow{#1}{\ifthenelse{\equal{#2}{}}{ }{#2 }\raisebox{.15ex}{\footnotesize \faExternalLink*}}}

% ---- header new cmd ---- %
\newcommand{\AND}{\unskip\cleaders\copy\ANDbox\hskip\wd\ANDbox\ignorespaces}\newsavebox\ANDbox\sbox\ANDbox{}

% ---- start document ---- %
\begin{document}
%   \placelastupdatedtext (that a legacy cmd replaced by footer)
    \begin{header}
        \fontsize{26 pt}{26 pt}
        \textbf{\BaseDecode{Q2zDqW1lbnQgVEFJTExJRVo=}}
        
        \vspace{0.3 cm}

        \normalsize
%       Note : comment out where it says color GRAY to display sensitive information
        \colorbox{gray}
        {\mbox{\hrefWithoutArrow{\BaseDecode{dGVsOiszMzY0MzgzMDQwMw==}}{
        \textcolor{gray}
        {{\footnotesize\faPhone*}\hspace*{0.13cm}\BaseDecode{MDYgNDMgODMgMDQgMDM=}}}}}
        \kern 0.25 cm
        \AND
        \kern 0.25 cm
        \colorbox{gray}
        {\mbox{\hrefWithoutArrow{\BaseDecode{bWFpbHRvOmN0YWlsbGllekBjdGEubGk=}}{
        \textcolor{gray}
        {{\footnotesize\faEnvelope}\hspace*{0.13cm}\BaseDecode{Y3RhaWxsaWV6QGN0YS5saQ==}}}}}
        \kern 0.25 cm
        \AND
        \kern 0.25 cm
        \colorbox{gray}
        {\mbox{\hrefWithoutArrow{https://osm.org/go/b~vJ1Ch-?relation=2753623}{
        \textcolor{gray}
        {{\footnotesize\faMapMarker*}\hspace*{0.13cm}\BaseDecode{UGF1ICg2NCk=}}}}}
        \kern 0.25 cm
        \AND
        \kern 0.25 cm
        \mbox{{{\footnotesize\faCarSide}\hspace*{0.13cm}Permis B}}
        \kern 0.25 cm
        \AND
        \kern 0.25 cm
        \mbox{{{\footnotesize\faBirthdayCake}\hspace*{0.13cm}23 ans}}
    \end{header}

    \vspace{0.3 cm - 0.3 cm}

    % === Diplômes et formations === %

    \section{Diplômes, Formations}

        % === BTS === %
        
        \begin{threecolentry}{\vspace{.4 mm}\textbf{BTS}}{
            Sept. 2019 $ \rightharpoonup $ Juil. 2021 \\
            Lycée St-John-Perse \\
            Pau (64)
        }
            \textbf{SIO} option solutions d'infrastructure, systèmes et réseaux
            \begin{highlights}
		          \item Réseau : Asterisk, GLPI (Fusion, OCS), HAProxy, IPTable, LAMP, MicroTik, NextCloud, Proxmox, Radius
		        \item Développement : Bash, Python, SQL
            \end{highlights}
        \end{threecolentry}

        \vspace{0.2 cm}

        % === BAC PRO === %

        \begin{threecolentry}{\textbf{BAC Pro}}{
            Sept. 2016 $ \rightharpoonup $ Juin 2019 \\
            Lycée Guynemer \\
            Oloron-Ste-Marie (64)
        }
            \textbf{SEN} option réseaux informatiques et systèmes communicants
            \begin{highlights}
                \item Réseau : bases (OSI, TCP/IP, IPv4-v6, DNS, DHCP, routage, VPN, HTTP-S, S-FTP, VLAN), switch (Cisco, Dlink, HP) Windows Server, Wireshark
                \item  Développement : Batch, C++
            \end{highlights}
        \end{threecolentry}

    % === Expériences Professionnelles === %
    
    \section{Expériences Professionnelles}

        % === Tibco Services === %
        
        \begin{twocolentry}{
            Jan. 2022 $ \rightharpoonup $ Oct. 2025 \\
            CDI \\
            Pau
        }
            \textbf{Tibco Services}, Technicien Retail n+2 itinérant 
            \begin{highlights}
                \item Intervention \href{https://w.wiki/D3ur}{IMAC} \& \href{https://w.wiki/D3uh}{MCO} et projets secteur 64-65 pour: BPCE, CAPG, Covea, Generix, HubOne, HSBC, InitSys, Macif, Nickel, OBS, Ortec, PMU, RCBT, Sephora, Tessi, THOM, U TECH, Yves Rocher...
                \item Tâches: Assistance user, déploiement, logistique, masterisation, remplacement (AP, bureautique, firewall, scan chèque, switch), retail (douchette, TPE, TPV...).
            \end{highlights}
        \end{twocolentry}

        \vspace{0.2 cm} % seeing for reduce this 

        % === DEFI Informatique === %

        \begin{twocolentry}{
            Juil. 2021 $ \rightharpoonup $ Jan. 2022 \\
            CDD \\
            Abos (64)
        }
            \textbf{DEFI Informatique}, Support Technique n+2 et supervision 
            \begin{highlights}
                \item BDD (MS Access), support d'un PGI (Gescof), Mail (MX, DMARC, SPF),
                Carbonite, scripting, tickets (logiciel interne), OVH, Zabbix.
            \end{highlights}
        \end{twocolentry}

        \vspace{0.2 cm}

        % === Stages BTS === %

        \begin{twocolentry}{
            4 et 7 semaines \\
            2 Stages \\
            Serres-Castet (64)
        }
            \textbf{Crédit Agricole}, Stagiaire BTS \textbf{\textit{S}}ervice \textbf{\textit{I}}nformatique aux \textbf{\textit{O}}rganisations
            \begin{highlights}
                \item Du 22/02/21 au 09/04/21 : Créations documentations, dépannage n+2 agence (bureautique, PLV, scanner chèques, etc.), enrôlement phone, tab (airwatch), masterisation postes, scripts PowerShell.
                \item Du 25/05/20 au 26/06/20 : Analyse d’export csv et json avec Excel, création de script batch, deploiement de poste (marimba), depannage salle de réunion et conférence (écran, Jabra, projecteur).
            \end{highlights}
        \end{twocolentry}

        \vspace{0.2 cm}

        % === Stages BAC PRO === %
        
        \begin{twocolentry}{
            25 semaines \\
            6 Stages \\
            \approx Pau
        }
            \textbf{Divers}, Stagiaire BAC PRO \textbf{\textit{S}}ystème \textbf{\textit{É}}lectronique \textbf{\textit{N}}umérique
            \begin{highlights}
                \item Scopelec, VBI, le Département 64, Disi (DGFIP), Ville de Pau (service feux tricolores), Emmaüs : aide, dépannages installations, principalement hardware.
            \end{highlights}
        \end{twocolentry}

    % === Compétences acquises === %
    
    \section{Compétences Acquises}
        
        \begin{onecolentry}
            \textbf{Langages:} Assembleur, batch, CSS, English B1 (intermediate), JavaScript, {\LaTeX}, PowerShell (en cours)
        \end{onecolentry}

        \vspace{0.2 cm}

        \begin{onecolentry}
            \textbf{Logiciels:} Docker, DSM, Git, GrapheneOS, G Workspace, IDA Pro, Inkscape, Office 365, VScode
        \end{onecolentry}

    % === Centre d'intérêt === %

    \section{Centre d'Intérêt}
        \begin{onecolentry}
            Cyclisme, gaming, informatique (programmation \& réseau), musique.
        \end{onecolentry}
    
\end{document}
