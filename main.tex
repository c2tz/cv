 \documentclass[11pt, a4paper]{article}
% Packages:
\usepackage[
    ignoreheadfoot, % set margins without considering header and footer
    top=1.0 cm, % seperation between body and page edge from the top
    bottom=2.0 cm, % seperation between body and page edge from the bottom
    left=2.0 cm, % seperation between body and page edge from the left
    right=2.0 cm, % seperation between body and page edge from the right
    footskip=1.0 cm, % seperation between body and footer
    % showframe % for debugging 
]{geometry} % for adjusting page geometry
\usepackage[explicit]{titlesec} % for customizing section titles
\usepackage{tabularx} % for making tables with fixed width columns
\usepackage{array} % tabularx requires this
\usepackage[dvipsnames]{xcolor} % for coloring text
\usepackage[svgnames]{xcolor} % for test
\definecolor{primaryColor}{RGB}{0, 79, 144} % define primary color
\usepackage{enumitem} % for customizing lists
\usepackage{fontawesome5} % for using icons
\usepackage{amsmath} % for math
\usepackage[
    pdftitle={CV - TAILLIEZ Clément},
    pdfauthor={TAILLIEZ Clément},
    pdfcreator={TAILLIEZ Clément},
    colorlinks=true,
    urlcolor=primaryColor
]{hyperref} % for links, metadata and bookmarks
\usepackage[pscoord]{eso-pic} % for floating text on the page
\usepackage{calc} % for calculating lengths
\usepackage{bookmark} % for bookmarks
\usepackage{lastpage} % for getting the total number of pages
\usepackage{changepage} % for one column entries (adjustwidth environment)
\usepackage{paracol} % for two and three column entries
\usepackage{ifthen} % for conditional statements
\usepackage{needspace} % for avoiding page brake right after the section title
\usepackage{iftex} % check if engine is pdflatex, xetex or luatex
\usepackage{datetime2} % for info about updt in footer
% \usepackage{tikz} % not in use for now
\DTMsetstyle{default}
\DTMsetdatestyle{ddmmyyyy}
\DTMsetup{datesep=/,hourminsep=h,minsecsep=:}

% Ensure that generate pdf is machine readable/ATS parsable:
\ifPDFTeX
    \input{glyphtounicode}
    \pdfgentounicode=1
    \usepackage[T1]{fontenc}
    \usepackage[utf8]{inputenc}
    \usepackage{lmodern}
\fi

% Use Roboto as default font
\usepackage[sfdefault]{roboto}

% Some settings:
\AtBeginEnvironment{adjustwidth}{\partopsep0pt} % remove space before adjustwidth environment
\pagestyle{empty} % no header or footer
\setcounter{secnumdepth}{0} % no section numbering
\setlength{\parindent}{0pt} % no indentation
\setlength{\topskip}{0pt} % no top skip
\setlength{\columnsep}{0.15cm} % set column seperation
\makeatletter
\let\ps@customFooterStyle\ps@plain % Copy the plain style to customFooterStyle
\patchcmd{\ps@customFooterStyle}{\thepage}{
    \color{darkgray}\begin{minipage}{\textwidth}
        \centering % Center the text on minipage
        \textrm{\footnotesize{Last update :}}
        \textrm{\footnotesize{\textit\date{\today} at \DTMcurrenttime{} ·}}
        \textrm{\footnotesize{\hrefWithoutArrow{https://github.com/c2tz/cv}{source code} [MIT License]}}
        \textrm{\footnotesize · p.\thepage{} on \pageref*{LastPage}}
    \end{minipage}
}{} {}
\makeatother
\pagestyle{customFooterStyle}

\titleformat{\section}{
    % avoid page breaking right after the section title
    \needspace{4\baselineskip}
    % make the font size of the section title large and color it with the primary color
    \Large\color{primaryColor}
}{
}{
}{
    % print bold title, give 0.15 cm space and draw a line of 0.8 pt thickness
    % from the end of the title to the end of the body
    \textbf{#1}\hspace{0.15cm}\titlerule[0.8pt]\hspace{-0.1cm}
}[] % section title formatting

\titlespacing{\section}{
    % left space:
    -1pt
}{
    % top space:
    0.3 cm
}{
    % bottom space:
    0.2 cm
} % section title spacing

% \renewcommand\labelitemi{$\vcenter{\hbox{\small$\bullet$}}$} % custom bullet points
\newenvironment{highlights}{
    \begin{itemize}[
        topsep=0.10 cm,
        parsep=0.10 cm,
        partopsep=0pt,
        itemsep=0pt,
        leftmargin=0.4 cm + 10pt
    ]
}{
    \end{itemize}
} % new environment for highlights

\newenvironment{highlightsforbulletentries}{
    \begin{itemize}[
        topsep=0.10 cm,
        parsep=0.10 cm,
        partopsep=0pt,
        itemsep=0pt,
        leftmargin=10pt
    ]
}{
    \end{itemize}
} % new environment for highlights for bullet entries


\newenvironment{onecolentry}{
    \begin{adjustwidth}{
        0.2 cm + 0.00001 cm
    }{
        0.2 cm + 0.00001 cm
    }
}{
    \end{adjustwidth}
} % new environment for one column entries

\newenvironment{twocolentry}[2][]{
    \onecolentry
    \def\secondColumn{#2}
    \setcolumnwidth{\fill, 4.5 cm}
    \begin{paracol}{2}
}{
    \switchcolumn \raggedleft \secondColumn
    \end{paracol}
    \endonecolentry
} % new environment for two column entries

\newenvironment{threecolentry}[3][]{
    \onecolentry
    \def\thirdColumn{#3}
    \setcolumnwidth{1 cm, \fill, 4.5 cm}
    \begin{paracol}{3}
    {\raggedright #2} \switchcolumn
}{
    \switchcolumn \raggedleft \thirdColumn
    \end{paracol}
    \endonecolentry
} % new environment for three column entries

\newenvironment{header}{
    \setlength{\topsep}{0pt}\par\kern\topsep\centering\color{primaryColor}\linespread{1.5}
}{
    \par\kern\topsep
}
\let\hrefWithoutArrow\href

% new command for external links:
\renewcommand{\href}[2]{\hrefWithoutArrow{#1}{\ifthenelse{\equal{#2}{}}{ }{#2 }\raisebox{.15ex}{\footnotesize \faExternalLink*}}}


\begin{document}
    \newcommand{\AND}{\unskip
        \cleaders\copy\ANDbox\hskip\wd\ANDbox
        \ignorespaces
    }
    \newsavebox\ANDbox
    \sbox\ANDbox{}
%    \placelastupdatedtext
    \begin{header}
        \fontsize{26 pt}{26 pt}
        \textbf{Clément Tailliez}

        \vspace{0.3 cm}

        \normalsize
        
        \colorbox{gray}{\mbox{\hrefWithoutArrow{tel:+33643830403}{\textcolor{gray}{\footnotesize\faPhone{} \hspace*{0.13cm}+33 6 43 83 04 03}}}}
        \kern 0.25 cm
        \AND
        \kern 0.25 cm
        \colorbox{gray}{\mbox{\hrefWithoutArrow{mailto:ctailliez@cta.li}{\textcolor{gray}{{\footnotesize\faEnvelope}\hspace*{0.13cm}ctailliez@cta.li}}}}
        \AND
        \kern 0.25 cm
        \AND
        \kern 0.25 cm
        \colorbox{gray}
        {\mbox{\hrefWithoutArrow{https://osm.org/go/b~vJaW6?relation=2195069}
        {\textcolor{gray}{{\footnotesize\faMapMarker*}\hspace*{0.13cm}{Artiguelouve (64)}}}}}
        \kern 0.25 cm
        \AND
        \kern 0.25 cm
        \mbox{{{\footnotesize\faCarSide}\hspace*{0.13cm}Permis B}}
        \kern 0.25 cm
        \AND
        \kern 0.25 cm
        \mbox{{{\footnotesize\faBirthdayCake}\hspace*{0.13cm}23 ans}}
    \end{header}

    \vspace{0.3 cm - 0.3 cm}

    % === Diplômes et formations === %

    \section{Diplômes, Formations}

        % === BTS === %
        
        \begin{threecolentry}{\vspace{.4 mm}\textbf{BTS}}{
            Sept. 2019 $ \rightharpoonup $ Juil 2021 \\
            Lycée St-John-Perse \\
            Pau (64)
        }
            \textbf{SIO} option solutions d'infrastructure, systèmes et réseaux
            \begin{highlights}
		          \item Réseau : Asterisk, GLPI (Fusion, OCS), HAProxy, IPTable, LAMP, MicroTik, NextCloud, PFSense, Proxmox, Radius, Zentyal 
		        \item Développement : Bash, PHP, PowerShell, Python, SQL, XML
            \end{highlights}
        \end{threecolentry}

        \vspace{0.2 cm}

        % === BAC PRO === %

        \begin{threecolentry}{\textbf{BAC Pro}}{
            Sept. 2016 $ \rightharpoonup $ Juin 2019 \\
            Lycée Guynemer \\
            Oloron-Ste-Marie (64)
        }
            \textbf{SEN} option réseaux informatiques et systèmes communicants
            \begin{highlights}
                \item Réseau : bases (binaire, DHCP, DNS, IP, masque, SSH, TLD, trames, routes, VPN...), switch (Cisco, Dlink, Nortel, HP) Windows Server, Wireshark
                \item  Développement : Batch, C++
            \end{highlights}
        \end{threecolentry}

    % === Expériences Professionnelles === %
    
    \section{Expériences Professionnelles}

        % === Tibco Services === %
        
        \begin{twocolentry}{
            Jan. 2022 $ \rightharpoonup $ Oct. 2025 \\
            CDI \\
            Pau
        }
            \textbf{Tibco Services}, Technicien Retail n+2 itinérant 
            \begin{highlights}
                \item Intervention \href{https://w.wiki/D3ur}{IMAC} \& \href{https://w.wiki/D3uh}{MCO} et projets secteur 64-65-32-40 pour: BPCE, CAPG, CCF, Covea, Generix, HubOne, InitSys, Macif, Nickel, Orange Business, Ortec, PMU, RCBT, Sephora, Tessi, THOM, U TECH, Yves Rocher...
                \item Tâches: Assistance utilisateurs, déploiement, logistique, masterisation postes, remplacement (AP, bureautique, firewall, scan chèque, switch), retail (douchette, TPE, TPV...).
            \end{highlights}
        \end{twocolentry}

        \vspace{0.2 cm} % seeing for reduce this 

        % === DEFI Informatique === %

        \begin{twocolentry}{
            Juil. 2021 $ \rightharpoonup $ Jan. 2022 \\
            CDD \\
            Abos (64)
        }
            \textbf{DEFI Informatique}, Support Technique n+2 et supervision 
            \begin{highlights}
                \item Debug BDD (MS Access), dépannage et installation d'un PGI (Gescof), Mail (MX, DMARC, SPF),
                % --- tricks for hyphénation --- %
                sauvegarde Carbonite, scripting, traitements tickets (logiciel interne), VPS OVH, Zabbix.
            \end{highlights}
        \end{twocolentry}

        \vspace{0.2 cm}

        % === Stages BTS === %

        \begin{twocolentry}{
            4 et 7 semaines \\
            2 Stages \\
            Serres-Castet (64)
        }
            \textbf{Crédit Agricole}, Stagiaire BTS \textbf{\textit{S}}ervice \textbf{\textit{I}}nformatique aux \textbf{\textit{O}}rganisations
            \begin{highlights}
                \item Du 22/02/21 au 09/04/21 : Créations documentations, dépannage n+2 agence (bureautique, PLV, scanner chèques, etc.), enrôlement phone, tab (airwatch), masterisation postes, scripts PowerShell.
                \item Du 25/05/20 au 26/06/20 : Analyse d’export csv et json avec Excel, création de script batch, deploiement de poste (marimba), depannage salle de réunion et conférence (écran, Jabra, projecteur).
            \end{highlights}
        \end{twocolentry}

        \vspace{0.2 cm}

        % === Stages BAC PRO === %
        
        \begin{twocolentry}{
            25 semaines \\
            6 Stages \\
            \approx Pau
        }
            \textbf{Divers}, Stagiaire BAC PRO \textbf{\textit{S}}ystème \textbf{\textit{É}}lectronique \textbf{\textit{N}}umérique
            \begin{highlights}
                \item Scopelec, VBI Jurançon, le Département 64, Disi (DGFIP), Ville de Pau (service feux tricolores), Emmaüs : aide, dépannages installations, principalement hardware.
            \end{highlights}
        \end{twocolentry}

    % === Compétences acquises === %
    
    \section{Compétences Acquises}
        
        \begin{onecolentry}
            \textbf{Langages:} Assembleur, CSS, English B1 (intermediate), JavaScript, \textit{\LaTeX}
        \end{onecolentry}

        \vspace{0.2 cm}

        \begin{onecolentry}
            \textbf{Logiciels:} Docker, DSM, Git, GrapheneOS, G Workspace, IDA Pro, Inkscape, Office 365, VScode
        \end{onecolentry}

    % === Centre d'intérêt === %

    \section{Centre d'Intérêt}
        \begin{onecolentry}
            Cyclisme, gaming, informatique (programmation \& réseau), musique.
        \end{onecolentry}
    
\end{document}
